\documentclass[aspectratio=169]{beamer}
\usepackage{xeCJK}
\setCJKmainfont{OPPO Sans} % 设置为Mac电脑最常见的字体

% =================== 主题设置 ====================
% 选择主题:Madrid。你也可以尝试其他主题如 Copenhagen、Berlin 等。
\usetheme{Madrid}

% 可选的其他主题定制:
% \usecolortheme{seahorse}    % 颜色主题:seahorse、dolphin、whale 等
% \usefonttheme{professionalfonts}  % 字体主题
% \useinnertheme{circles}     % 内部元素主题(如列表符号样式)
% \useoutertheme{infolines}   % 外部模板,控制页眉页脚等布局

% =================== 文档信息 ====================
\title{Beamer 全面语法讲解}         % 演示文稿标题
\subtitle{详细介绍 Beamer 的各种语法与功能} % 副标题,可选
\author{张三 \& 李四}               % 作者,可以写多个名字
\institute{某某大学 \\ 某某研究所}    % 所属机构,可以换成你所在的单位或学校
\date{\today}                   % 使用系统当前日期

% =================== 引用宏包 ====================
% 引入插入图片、数学公式、绘图等常用宏包
\usepackage{graphicx}     % 用于插入图片
\usepackage{amsmath, amssymb}  % 用于数学公式排版
\usepackage{tikz}         % 用于绘制流程图、示意图等
\usetikzlibrary{positioning} % 加载定位库以支持 below=of 语法
% 根据需要还可以加载其他宏包,比如 listings(代码显示)、xcolor(颜色支持)等

\begin{document}

% ================ 标题页 ======================
\begin{frame}
    \titlepage  
    % 利用 \titlepage 命令自动生成标题页,展示标题、作者、机构和日期等信息
\end{frame}

% ================ 目录页 ======================
\begin{frame}
    \frametitle{目录}
    % 自动生成目录:依据下面的 \section 与 \subsection 自动构建目录
    \tableofcontents
\end{frame}

% ================ 第一部分:Beamer 基础简介 ======================
\section{Beamer 基础简介}
\begin{frame}
    \frametitle{什么是 Beamer?}
    % Beamer 是一个基于 LaTeX 的演示文稿系统,其特点包括:
    \begin{itemize}
        \item 高质量的排版,尤其适合排版数学公式与复杂图表
        \item 丰富的主题与模板,可以定制化演示文稿外观
        \item 支持动画与动态效果,让讲解更富有节奏感
        \item 易于集成代码、图片和表格等内容
    \end{itemize}
\end{frame}

% ================ 第二部分:列表与动画效果 ======================
\section{列表与动画效果}
\begin{frame}
    \frametitle{项目符号列表示例}
    % 使用 itemize 环境生成无序列表,现一次性显示全部内容
    \begin{itemize}
        \item 第一要点
        \item 第二要点
        \item 第三要点
    \end{itemize}
\end{frame}

\begin{frame}
    \frametitle{有序列表示例}
    % 使用 enumerate 环境生成有序列表,现一次性显示全部内容
    \begin{enumerate}
        \item 第一项
        \item 第二项
        \item 第三项
    \end{enumerate}
\end{frame}

% ================ 第三部分:Block 环境的使用 ======================
\section{Block 环境的使用}
\begin{frame}
    \frametitle{Block 环境介绍}
    % Block 环境用于突出显示某一段重要信息或分块内容
    \begin{block}{普通 Block}
        这是一个普通的 Block,用于解释某个概念或提供说明文字。
    \end{block}
    
    \begin{alertblock}{警告 Block}
        这是一个警告 Block,适用于强调注意和错误提示信息。
    \end{alertblock}
    
    \begin{exampleblock}{示例 Block}
        这是一个示例 Block,用来展示具体的例子或案例分析。
    \end{exampleblock}
\end{frame}

% ================ 第四部分:数学公式与自定义命令 ======================
\section{数学公式与自定义命令}
\begin{frame}
    \frametitle{数学公式示例}
    % 行内公式示例:使用美元符号包裹公式内容
    爱因斯坦质能关系: $E = mc^2$。
    % 独立公式:使用 display math 环境展示更复杂的公式
    \[
    \sum_{n=1}^{\infty} \frac{1}{n^2} = \frac{\pi^2}{6}
    \]
    % 定义自定义命令:下面定义一个表示复数集的命令,并在公式中使用
    \newcommand{\Real}{\mathbb{R}}
    这里定义了实数集合:$\Real$。
\end{frame}

% ================ 第五部分:多栏布局示例 ======================
\section{多栏布局示例}
\begin{frame}
    \frametitle{两栏布局示例}
    % 使用 columns 环境生成多栏排版,使得内容可以左右分栏显示
    \begin{columns}[T] % [T] 表示顶部对齐
        \column{0.48\textwidth}
            % 左栏内容:展示列表
            \textbf{左栏:}
            \begin{itemize}
                \item 条目 1
                \item 条目 2
                \item 条目 3
            \end{itemize}
        \column{0.48\textwidth}
            % 右栏内容:插入图片展示
            \textbf{右栏:}
            \begin{figure}
                \includegraphics[width=\textwidth]{example-image} % 请替换为正确的图片文件名
                \caption{示例图片}
            \end{figure}
    \end{columns}
\end{frame}

% ================ 第六部分:图片与表格 ======================
\section{图片与表格}
\begin{frame}
    \frametitle{插入图片示例}
    % 插入图片的基本用法,与标准 LaTeX 用法一致
    \begin{figure}
        \centering
        \includegraphics[width=0.7\textwidth]{example-image}
        \caption{这是一个示例图片}
    \end{figure}
\end{frame}

\begin{frame}
    \frametitle{插入表格示例}
    % 使用 tabular 环境插入简单表格,注意定义列格式和表格线条
    \begin{center}
    \begin{tabular}{|c|c|c|}
        \hline
        列 1 & 列 2 & 列 3 \\
        \hline
        数据 A & 数据 B & 数据 C \\
        \hline
        数据 D & 数据 E & 数据 F \\
        \hline
    \end{tabular}
    \end{center}
\end{frame}

% ================ 第七部分:动态效果与动画 ======================
\section{动态效果与动画}
\begin{frame}[fragile]
    \frametitle{动态效果与 Overlay}
    % 下面的列表直接显示全部内容:
    \begin{itemize}
        \item 动画第一步
        \item 动画第二步
        \item 动画第三步
    \end{itemize}
    此外,可使用 \verb|\only|、\verb|\visible|、\verb|\invisible| 等命令精细控制内容显示。
\end{frame}

% ================ 第八部分:使用 TikZ 绘图 ======================
\section{使用 TikZ 绘图}
\begin{frame}
    \frametitle{TikZ 绘图示例}
    % 使用 TikZ 宏包绘制简单流程图
    \begin{tikzpicture}[node distance=1.5cm, auto]
        % 定义节点
        \node[draw, rectangle] (start) {开始};
        \node[draw, rectangle, below=of start] (process) {处理过程};
        \node[draw, rectangle, below=of process] (end) {结束};
        % 连接节点:使用箭头连接
        \draw[->] (start) -- (process);
        \draw[->] (process) -- (end);
    \end{tikzpicture}
\end{frame}

% ================ 第九部分:定制页眉页脚 ======================
\section{定制页眉页脚}
\begin{frame}[fragile]
    \frametitle{自定义页眉页脚示例}
    % 利用 \setbeamertemplate 自定义页脚,这里将作者、日期和页码显示在页面底部
    \setbeamertemplate{footline}{
        \leavevmode%
        \hbox{%
            \begin{beamercolorbox}[wd=0.5\paperwidth,ht=2.5ex,dp=1ex,center]{author in head/foot}%
                \usebeamerfont{author in head/foot}\insertshortauthor
            \end{beamercolorbox}%
            \begin{beamercolorbox}[wd=0.5\paperwidth,ht=2.5ex,dp=1ex,center]{date in head/foot}%
                \usebeamerfont{date in head/foot}\insertshortdate\,\insertframenumber{} / \inserttotalframenumber
            \end{beamercolorbox}%
        }
    }
    % 注意:自定义页眉页脚的设置可以放在导言区,或者在需要时局部修改
\end{frame}

% ================ 第十部分:高级定制 ======================
\section{高级定制}
\begin{frame}[fragile]
    \frametitle{修改标题页模板示例}
    % 通过 \setbeamertemplate 自定义标题页的布局
    \setbeamertemplate{title page}{
        \vbox{}
        \vfill
        \begin{center}
            {\usebeamerfont{title}\inserttitle\par}
            \vskip0.5em
            {\usebeamerfont{subtitle}\insertsubtitle\par}
            \vskip1em
            {\usebeamerfont{author}\insertauthor\par}
            \vskip1em
            {\usebeamerfont{institute}\insertinstitute\par}
            \vskip1em
            {\usebeamerfont{date}\insertdate\par}
        \end{center}
        \vfill
    }
    % 调用 \titlepage 命令后即可显示自定义的标题页
    此处展示了如何自定义标题页,使其样式和排版满足个性化需求。
\end{frame}

% ================ 第十一部分:参考文献 ======================
\section{参考文献}
\begin{frame}[allowframebreaks]
    \frametitle{参考文献示例}
    % 如果你使用 BibTeX 或 BibLaTeX 管理参考文献,可在这里插入相应代码
    % 示例(需要准备 references.bib 文件):
    % \bibliographystyle{plain}
    % \bibliography{references}
    %
    % allowframebreaks 允许内容在多个页面上分割显示
    这里是参考文献部分,实际使用时请根据你的文献数据库进行配置。
\end{frame}

% ================ 第十二部分:结束页 ======================
\section{结束}
\begin{frame}
    \frametitle{感谢聆听}
    % 演示文稿的最后一个页面,用于致谢或者总结
    感谢大家的聆听!欢迎提问讨论!
\end{frame}

\end{document}
